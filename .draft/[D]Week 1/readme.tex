\documentclass[11pt]{article}
\usepackage[top=1cm, bottom=2cm, left=1cm, right=1cm]{geometry}
\usepackage{ctex}
\usepackage[linesnumbered,ruled]{algorithm2e}
\usepackage{amsthm,amsmath,amssymb}
\usepackage[colorlinks=true,linkcolor=blue]{hyperref}
\usepackage{listings}
\usepackage{xcolor,xparse}
\usepackage{realboxes}
\usepackage{graphics}
\usepackage{graphicx}
\usepackage{mathrsfs}
\usepackage{wrapfig}
\usepackage{subfigure}
\usepackage{pifont}

\definecolor{cmdbg}{rgb}{0.9,0.9,0.9}
\lstset{%
	basicstyle=\ttfamily,
	breaklines = true,
	backgroundcolor=\color{cmdbg},
}
\DeclareDocumentCommand{\ccmd}{v}{% 参数 v 表示工作方法类似于 \verb
    \Colorbox{cmdbg}{\csname lstinline\endcsname!#1!}%
}

% \makeatletter
% \newenvironment{breakablealgorithm}
%   {% 这里是环境的开始部分 \begin{breakablealgorithm}
%    \begin{center} % 居中显示
%      \refstepcounter{algorithm}% 增加算法计数器
%      \hrule height.8pt depth0pt \kern2pt% 插入一条 0.8pt 高的水平线和 2pt 间距
%      \renewcommand{\caption}[2][\relax]{% 重定义 \caption 命令,接收参数依次为目录中的标题(可选)与正文中的标题
%        {\raggedright\textbf{\ALG@name~\thealgorithm} ##2\par}% 显示算法编号和标题
%        \ifx\relax##1\relax % 如果没有传递可选参数 #1
%          \addcontentsline{loa}{algorithm}{\protect\numberline{\thealgorithm}##2}% 默认将正文中的标题直接添加到目录中
%        \else % 如果传递了可选参数 #1
%          \addcontentsline{loa}{algorithm}{\protect\numberline{\thealgorithm}##1}% 将自定义的目录标题添加到目录中
%        \fi
%        \kern2pt\hrule\kern2pt % 再插入一条水平线
%      }
%   }{% 这里是环境的结束部分 \end{breakablealgorithm}
%      \kern2pt\hrule\relax % 再次插入一条水平线,结束
%    \end{center}
%   }
% \makeatother


\author{杨远青 22300190015}
\title{计算物理作业1}

\begin{document}
\maketitle

\section{题目 1:五次幂丢番图方程}

\subsection{题目描述}
\noindent
Consider the Poisson equation:
\[
    \nabla^2 \varphi(x, y) = -\frac{\rho(x, y)}{\varepsilon_0}
\]
from electrostatics on a rectangular geometry with \(x \in [0, L_x]\) and \(y \in [0, L_y]\). Write a program that solves this equation using the relaxation method and test your program with the following cases:

\noindent
(a) \(\rho(x, y) = 0\), \(\varphi(0, y) = \varphi(L_x, y) = \varphi(x, 0) = 0\), \(\varphi(x, L_y) = 1 \, \text{V}\),
\(L_x = 1 \, \text{m}\), and \(L_y = 1.5 \, \text{m}\);

\noindent
(b) \(\frac{\rho(x, y)}{\varepsilon_0} = 1 \, \text{V/m}^2\), \(\varphi(0, y) = \varphi(L_x, y) = \varphi(x, 0) = \varphi(x, L_y) = 0\), and \(L_x = L_y = 1 \, \text{m}\).


\subsection{程序描述}

\subsection{伪代码}
Powered by \href{https://chatgpt.com/g/g-xJJAA2awf-latex-pseudocode-generator}{\LaTeX \ pseudocode generator}


\subsection{结果示例}

% \section{题目 2:24点问题}
% \incput{problem_2.tex}

\end{document}
