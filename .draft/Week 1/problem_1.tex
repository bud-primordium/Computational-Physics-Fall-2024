\subsection{题目描述}
Find all integer solutions to the \textbf{Diophantine equation} \( a^5 + b^5 + c^5 + d^5 = e^5 \) within the range \([0, 200]\).
\subsection{程序描述}

\subsection{伪代码}
的伪代码如下所示

\begin{algorithm}[H]
    \caption{Brute-force solution to the Diophantine equation}
    \KwIn{$N$: Integer (the upper bound, $N = 200$)}
    \KwOut{$solutions$: List of tuples $(a, b, c, d, e)$\tcp*{where $0 \leq a \leq b \leq c \leq d < e \leq N$}}
    \For{$a \gets 0$ \KwTo $N$}{
        \For{$b \gets a$ \KwTo $N$ }{
            \For{$c \gets b$ \KwTo $N$}{
                \For{$d \gets c$ \KwTo $N$}{
                    \For{$e \gets d + 1$ \KwTo $N$}{
                        \If{$a^5 + b^5 + c^5 + d^5 = e^5$\tcp*{Check if the tuple is a solution}}{
                            
                            solutions.append($(a, b, c, d, e)$)
                            \tcp*{Store the solution tuple}
                        }
                    }
                }
            }
        }
    }
    \Return $solutions$\tcp*[r]{Return the list of solution tuples}
\end{algorithm}
\begin{algorithm}
    \KwData{Input data}
    \KwResult{Output result}
    \For{i = 1 to n}{
        Process data\; \tcc*[l]{This is a comment using tcc}
        Process more data\; \tcp{Another comment using tcp}
    }
    \end{algorithm}
    



\begin{algorithm}[H]
    \caption{Mod-30 trick solution to the Diophantine equation}
    \KwIn{$N$: 整数上限($N = 200$)}
    \KwOut{所有满足 $a^5 + b^5 + c^5 + d^5 = e^5$ 且 $0 \leq a \leq b \leq c \leq d < e \leq N$ 的整数解 $(a, b, c, d, e)$}
    \For{$a \gets 0$ \KwTo $N$}{
        \For{$b \gets a$ \KwTo $N$}{
            \For{$c \gets b$ \KwTo $N$}{
                \For{$d \gets c$ \KwTo $N$}{
                    \For{$e \gets d + 1$ \KwTo $N$}{
                        \If{$a^5 + b^5 + c^5 + d^5 = e^5$}{
                            \KwResult{$(a, b, c, d, e)$}
                        }
                    }
                }
            }
        }
    }
\end{algorithm}




\subsection{输入输出实例}
对于本程序,首先需要用户输入电路中六个电阻$(r_s,r_a,r_x,r_1,r_2,r_3)$的数值,通过这些电阻值写出增广矩阵$\boldsymbol{R}|\boldsymbol{v}$,将该增广矩阵带入高斯消去法中即可求得电流$\boldsymbol{i}$,等效电阻$r_e=v_0/i_1$。下列表格为在相应输入电阻下的运算结果
