\documentclass[11pt]{article}
\usepackage[top=1cm, bottom=2cm, left=1cm, right=1cm]{geometry}
\usepackage{ctex}
\usepackage[linesnumbered,ruled]{algorithm2e}
\usepackage{amsthm,amsmath,amssymb}
\usepackage[colorlinks=true,linkcolor=blue]{hyperref}
\usepackage{listings}
\usepackage{xcolor,xparse}
\usepackage{realboxes}
\usepackage{mathrsfs}
\usepackage{wrapfig}
\usepackage{subfigure}
\usepackage{forest}
\usepackage{pifont}
\usepackage{ulem}
\usepackage{bm}

\SetKw{Print}{print}
\SetKw{Read}{read}
\SetKw{Allocate}{allocate}
\SetKw{Deallocate}{deallocate}
\SetKw{Call}{call}
\SetKw{True}{true}
\SetKw{False}{false}
\SetKw{Continue}{continue}
\SetKw{Break}{break}
\SetKw{Write}{write}

\definecolor{cmdbg}{rgb}{0.9,0.9,0.9}
\lstset{%
	basicstyle=\ttfamily,
	breaklines = true,
	backgroundcolor=\color{cmdbg},
}
\DeclareDocumentCommand{\ccmd}{v}{% 参数 v 表示工作方法类似于 \verb
    \Colorbox{cmdbg}{\csname lstinline\endcsname!#1!}%
}


\author{杨远青 22300190015}
\title{计算物理作业3}

\begin{document}
\maketitle
\textit{远方来朋,喜;假期俱至,悦。}

\section{\texorpdfstring{\sout{题目 1:高斯消元法的时间复杂度分析}}{题目 1}}
\subsection{题目描述}
Prove that the time complexity of Gaussian elimination algorithm is $\mathcal{O}(n^3)$.
\subsection{证明}
Gaussian消元法,此处特指\textit{Forward Elimination} \& \textit{Backward Substitution}法,而不是最古老的Gaussian-Jordan消元法(用于求逆的某浪漫主义教学算法),在大多数情况下的表现,并不如兼具精确度与效率的$\bm{LU}$分解法,但一些思想被嵌入后者与适用于更大规模矩阵求解的各类迭代算法中,因此仍有必要对其进行分析。

先考虑\textit{Forward Elimination}的时间复杂度,即通过初等行变换将原本的增广矩阵
$\bm{(A \, | \, b)}$
\[
	\left[\begin{array}{cccc|c}a_{11}&a_{12}&\cdots&a_{1n}&b_{1}\\a_{21}&a_{22}&\cdots&a_{2n}&b_{2}\\\vdots&\vdots&\ddots&\vdots&\vdots\\a_{n 1}&a_{n 2}&\cdots&a_{n n}&b_{n}\end{array}\right]
\]
上三角化为$\bm{U}$。暂不考虑\textit{Pivot}步骤可能带来的交换操作,尽管这对于提升数值稳定性非常重要。考虑第$1$列的第$2$至$n$行,每一行需要先计算系数$a_{i1}/a_{11}$,再进行$n$次乘法与$n$次减法(各行首元素直接设为$0$,不计入乘减法操作,但要考虑最右侧$b$的元素),故第$1$列的消元操作数为$(n-1)(2n+1)$,递推可知,第$i$步便是对$(n-i+1)\times(n-i+1)$子矩阵的消元,迭代操作数为$(n-i)(2n-2i+3)$,总操作数为
\[
	T_F(n) =
	\sum_{i=1}^{n-1} (2n-2i+3)(n-i) = 2\sum_{i=1}^{n-1} (n-i)(n-i)+3\sum_{i=1}^{n-1} (n-i) = \frac{4n^3+3n^2-7n}{6}.
\]
再考虑\textit{Backward Substitution}的时间复杂度,当我们消元得到一个$n \times n$的上三角矩阵$\bm{U}$
\[
	\left[\begin{array}{cccc|c}a_{11}'&a_{12}'&\cdots&a_{1 n}'&b_{1}'\\0&a_{22}'&\cdots&a_{2 n}'&b_{2}'\\\vdots&\vdots&\ddots&\vdots&\vdots\\0&0&\cdots&a_{n n}'&b_{n}'\end{array}\right]
\]
之后,需要从最后一行开始,逐行求解
\[
	x_i=\frac{1}{a_{i i}'}\left(b_i'-\sum_{j=i+1}^ka_{i j}' x_j\right).
\]
每一行涉及的四则运算(我们非常流氓地忽视除法的独特地位,理论上这需要基于牛顿迭代的现代方法进行特殊处理)为$(n-i)$次乘法与$(n-i)$次减法,再进行$1$次除法,故每行的操作数为$[2(n-i)+1]$,总操作数为
\[
	T_B(n)=\sum_{i=1}^n \ [2(n-i)+1]=2\sum_{i=1}^n (n-i)+n=n^2.
\]
故Gaussian消元法的总操作数为
\[
	T(n)=T_F(n)+T_B(n)=\frac{4n^3+3n^2-7n}{6}+n^2=\frac{4n^3+9n^2-7n}{6}.
\]
其中有除法$n(n+1)/2$次,乘法与减法各$n(n-1)(2n+5)/6$次,故
\[
	\boxed{T(n) = \mathcal{O}(n^3)}
\]

伙计,这听起来一点也不酷,怎么到头来还是和求逆矩阵一样是$\mathcal{O}(n^3)$?但如果我们将\textit{Substitution}的思想嵌入到$\bm{LU}$分解法\footnote{详见\textit{Numerical Recipes $\S 2.4$}},对一些特定情形,譬如三对角矩阵的回代操作可以从$\mathcal{O}(n^2)$优化到$\mathcal{O}(n)$,且对于不同的待解向量$\bm{b}$,我们的圣遗物$\bm{L}$和$\bm{U}$可以被重复利用,这听上去还是不错的!

如果想和理论计算机科学家一样,执着于对$\mathcal{O}(n^3)$的优化:Strassen的构造可以帮你将指数因子优化到$\mathcal{O}(n^{\log_2 7})$,即$\omega  = \log_2 7 \approx 2.8074$\footnote{有个直观而有趣的讨论,详见\textit{Numerical Recipes $\S 2.11$}},采用Coppersmith–Winograd矩阵乘法可以优化到$\omega \le 2.3755$\footnote{$\omega < 2.404$的一种证明,参见\ \href{https://people.csail.mit.edu/virgi/6.890/lecture23.pdf}{\textit{MIT6.890 $\S 23$}}}.但这类小数点后的“用力过度”不是我们的菜,有时候反倒是滥用主定理,即它们所需的天文数字规模$N\times N$的矩阵来临时,我们早该另觅出路,比如考虑使用Jacobi等迭代法。

\setcounter{section}{0}
\vspace{1em}
\textit{公元二〇二四年九月二十四日,午时三刻,于HGX106室,惊闻徐夫子欲改弦更张,悲哉!}
\vspace{-1em}
\section{\texorpdfstring{题目 1:$\bm{LU}$分解法的时间复杂度分析}{题目 1:LU分解法的时间复杂度分析}}
\subsection{题目描述}
Prove that the time complexity of $\bm{LU}$ decomposition algorithm is $\mathcal{O}(n^3)$.
\subsection{证明}
$\bm{LU}$分解法的第一步是将系数矩阵$\bm{A}$分解为一个下三角矩阵$\bm{L}$和一个上三角矩阵$\bm{U}$:
\[
    \begin{bmatrix}
        a_{11} & a_{12} & \cdots & a_{1n} \\
        a_{21} & a_{22} & \cdots & a_{2n} \\
        \vdots & \vdots & \ddots & \vdots \\
        a_{n1} & a_{n2} & \cdots & a_{nn}
    \end{bmatrix}
    =
    \begin{bmatrix}
        l_{11} & 0      & \cdots & 0      \\
        l_{21} & l_{22} & \cdots & 0      \\
        \vdots & \vdots & \ddots & \vdots \\
        l_{n1} & l_{n2} & \cdots & l_{nn}
    \end{bmatrix}
    \begin{bmatrix}
        1      & u_{12} & \cdots & u_{1n} \\
        0      & 1      & \cdots & u_{2n} \\
        \vdots & \vdots & \ddots & \vdots \\
        0      & 0      & \cdots & 1
    \end{bmatrix}.
\]
这一步常采用Crout方法实现,即在每一轮中,我们先计算$\bm{L}$的第$k$列元素$l_{ik}$,
\[
    l_{ik} = a_{ik} - \sum_{s=1}^{k-1} l_{is} u_{sk}, \quad \ i = k, k+1, \dots, n.
\]
每一个$l_{ik}$的计算涉及$k-1$次乘法和$k-1$次减法,共有$(n-k+1)$个$l_{ik}$需要计算;再计算$\bm{U}$的第$k$行元素$u_{kj}$,
\[
    u_{kj} = \frac{1}{l_{kk}} \left( a_{kj} - \sum_{s=1}^{k-1} l_{ks} u_{sj} \right), \quad \ j = k+1, k+2, \dots, n.
\]
相比$l_{ik}$的计算多了一次除法,共有$(n-k)$个$u_{kj}$需要计算,故第$k$轮的操作数为
\[
    (n-k+1)\cdot(2k-2)+(n-k)\cdot(2k-1) = -4k^2 + (4n+5)k -3n -2.
\]
因此,分解步骤的总操作数为
\[
    T_c(n) =\sum_{k=1}^{n} \ [-4k^2 + (4n+5)k -3n -2]                                                 \\
    = -4 \cdot \frac{n(n+1)(2n+1)}{6} + (4n+5) \cdot \frac{n(n+1)}{2} - (3n+2) \cdot n\\
    = \frac{4n^3 - 3n^2 - n}{6}.
\]

再考虑回代步骤的操作数,即用分解得到的$\bm{L}$和$\bm{U}$求解方程组$\bm{Ax} = \bm{b}$。首先求解$\bm{Ly} = \bm{b}$,即
\[
    \begin{bmatrix}
        l_{11} & 0      & \cdots & 0      \\
        l_{21} & l_{22} & \cdots & 0      \\
        \vdots & \vdots & \ddots & \vdots \\
        l_{n1} & l_{n2} & \cdots & l_{nn}
    \end{bmatrix}
    \cdot
    \begin{bmatrix}
        y_1 \\ y_2 \\ \vdots \\ y_n
    \end{bmatrix}
    =
    \begin{bmatrix}
        b_1 \\ b_2 \\ \vdots \\ b_n
    \end{bmatrix}.
\]
这实质上是从第一行开始的\textit{Forward Substitution},即
\[
    y_i=\frac{1}{l_{i i}}\left(b_i-\sum_{j=1}^{i-1} l_{i j} y_j\right).
\]
每一步有$1$次除法,$(i-1)$次乘法与$(i-1)$次减法;再求解$\bm{Ux} = \bm{y}$,即
\[
    \begin{bmatrix}
        1      & u_{12} & \cdots & u_{1n} \\
        0      & 1      & \cdots & u_{2n} \\
        \vdots & \vdots & \ddots & \vdots \\
        0      & 0      & \cdots & 1
    \end{bmatrix}
    \cdot
    \begin{bmatrix}
        x_1 \\ x_2 \\ \vdots \\ x_n
    \end{bmatrix}
    =
    \begin{bmatrix}
        y_1 \\ y_2 \\ \vdots \\ y_n
    \end{bmatrix},
\]
这实质上是从最后一行开始的\textit{Backward Substitution},即
\[
    x_i=\left(y_i-\sum_{j=i+1}^{n} u_{i j} x_j\right).
\]
每一步有$(n-i)$次乘法与$(n-i)$次减法,故回代步骤操作数为
\[
    T_s(n)=\sum_{i=1}^{n} \ [(2i-1)+(2n-2i)] = \sum_{i=1}^{n} (2n-1) = n(2n-1) = 2n^2 -n.
\]
因此,$\bm{LU}$分解法的总操作数为
\[
    T(n)=T_c(n)+T_s(n)=\frac{4n^3 -3n^2 -n}{6} + 2n^2 -n = \frac{4n^3 +9n^2 -7n}{6}.
\]
其中有除法$n(n+1)/2$次,乘法与减法各$n(n-1)(2n+5)/6$次,故
\[
    \boxed{T(n) = \mathcal{O}(n^3)}
\]
\textit{Amazing,居然与Gaussian消元法的各种操作数都相同!}

\section{题目 2:结合部分主元应用高斯消元法}
\subsection{题目描述}
\noindent
Solve the time-dependent Schrödinger equation using both the Crank–Nicolson scheme and a stable explicit scheme. Consider the one-dimensional case and test it by applying it to the problem of a square well with a Gaussian initial state coming in from the left.

\noindent
Hint: The Gaussian initial state could be expressed as:
\[
    \Psi(x, 0) = \sqrt{\frac{1}{\pi}} \exp\left[ i k_0 x - \frac{(x - \xi_0)^2}{2} \right].
\]


\subsection{程序描述}
本程序通过\texttt{Parameters}类管理参数,包括网格参数(时空坐标剖分)、势阱参数(宽度、深度与中心位置)以及初始波包参数(宽度、位置与动量)。定义了一个\texttt{SchrodingerSolver}求解器基类,包含了Crank–Nicolson解法与显式解法的接口,以及一些共用的方法,如检查输入参数是否满足Von Neumann稳定性条件。两个求解器\texttt{CrankNicolsonSolver}与\texttt{ExplicitSolver}继承自基类,分别实现了Crank–Nicolson解法与显式解法。本题求解的一维含时薛定谔方程
\[
    i\hbar \frac{\partial \psi(x,t)}{\partial t} = -\frac{\hbar^2}{2m} \frac{\partial^2 \psi(x,t)}{\partial x^2} + V(x) \psi(x,t)
\]在原子单位制$\hbar = m =1$下化简为
\[
    i \frac{\partial \psi}{\partial t} = -\frac{1}{2} \frac{\partial^2 \psi}{\partial x^2} + V(x) \psi
\]
离散化时记 $\psi_j^n$ 为在位置 $x_j = j \Delta x $和时间 $t^n = n \Delta t $处的波函数值。
\subsubsection{Crank-Nicolson算法}
类比二阶偏微分方程中的Gauss-Seidel迭代,隐式的Crank-Nicolson 算法对时间导数采用前向差分,对中心差分的空间导数和势能项取时间切片 \(n\) 和 \(n+1\) 的平均,即:
\[
    i \frac{\psi_j^{n+1} - \psi_j^n}{\Delta t} = -\frac{1}{4} \left( \frac{\psi_{j+1}^{n+1} - 2\psi_j^{n+1} + \psi_{j-1}^{n+1}}{(\Delta x)^2} + \frac{\psi_{j+1}^n - 2\psi_j^n + \psi_{j-1}^n}{(\Delta x)^2} \right) + \frac{1}{2} V_j \left( \psi_j^{n+1} + \psi_j^n \right)
\]
可以整理为半步演化形式
\[
    \left( 1 + i \frac{\Delta t}{2} \hat{H} \right) \psi^{n+1} = \left( 1 - i \frac{\Delta t}{2} \hat{H} \right) \psi^n
\]
其中哈密顿算符在每个时间切片上离散化为
\[
    \hat{H} \psi_j = -\frac{1}{2 (\Delta x)^2} \left( \psi_{j+1} - 2\psi_j + \psi_{j-1} \right) + V_j \psi_j
\]
实际代码实现中,构造了两个三对角矩阵,化方程为$\mathbf{A} \psi^{n+1} = \mathbf{B} \psi^n$,满足$\mathbf{A} = \mathbf{I} + i \frac{\Delta t}{2} \mathbf{H}\text{与}\mathbf{B} = \mathbf{I} - i \frac{\Delta t}{2} \mathbf{H}$,即
\[
    \mathbf{A}, \mathbf{B} =
    \begin{pmatrix}
        1 + 2\alpha \pm \frac{i \Delta t}{2} V_1 & \mp \alpha                               & 0                                        & \cdots     & 0                                            \\
        \mp \alpha                               & 1 + 2\alpha \pm \frac{i \Delta t}{2} V_2 & \mp \alpha                               & \ddots     & \vdots                                       \\
        0                                        & \mp \alpha                               & 1 + 2\alpha \pm \frac{i \Delta t}{2} V_3 & \ddots     & 0                                            \\
        \vdots                                   & \ddots                                   & \ddots                                   & \ddots     & \mp \alpha                                   \\
        0                                        & \cdots                                   & 0                                        & \mp \alpha & 1 + 2\alpha \pm \frac{i \Delta t}{2} V_{N_x} \\
    \end{pmatrix}
\]
其中$
    \alpha = \frac{i \Delta t}{4 (\Delta x)^2}
$,模长需满足Von Neumann稳定性条件。

在演化步骤中,三对角矩阵均使用 \texttt{CSC} 稀疏格式存储,并使用 \texttt{scipy.sparse.linalg.spsolve} 进行求解,以提高计算效率和节省内存。
\subsubsection{显式算法}
显式算法对时间导数与空间导数采用中心差分,但不对相邻切片的势能或者空间导数进行平均
\[
    i \frac{\psi_j^{n+1} - \psi_j^{n-1}}{2\Delta t} = -\frac{1}{2} \frac{\psi_{j+1}^n - 2\psi_j^n + \psi_{j-1}^n}{(\Delta x)^2} + V_j \psi_j^n
\]
故可以直接进行显式更新
\[
    \psi_j^{n+1} = \psi_j^{n-1} + \frac{i \Delta t}{\Delta x^2} \left( \psi_{j+1}^n + \psi_{j-1}^n - 2\psi_j^n \right) - 2i \Delta t V_j \psi_j^n
\]
化为矩阵形式即为
\[
    \psi^{n+1} = \psi^{n-1} + \frac{i \Delta t}{(\Delta x)^2} \mathbf{L} \psi^n - 2i \Delta t \mathbf{V} \psi^n
\]
其中$\mathbf{V}$即离散的势能项,动能项为拉普拉斯算符
\[
    \mathbf{L} =
    \begin{pmatrix}
        -2     & 1      & 0      & \cdots & 1      \\
        1      & -2     & 1      & \cdots & 0      \\
        0      & 1      & -2     & \cdots & 0      \\
        \vdots & \ddots & \ddots & \ddots & \vdots \\
        1      & 0      & 0      & 1      & -2
    \end{pmatrix}
\]
实际代码实现中没有显式定义$\mathbf{L}$,而是在每一步中借助\texttt{np.roll}函数(将数组在周期边界条件下顺次移动),直接计算了$\mathbf{L} \psi^n$。在第一步中因为没有$\psi^{n-1}$,故使用Crank-Nicolson算法进行第一步演化,之后都使用显式算法进行演化。
\subsection{伪代码}
Powered by \href{https://chatgpt.com/g/g-xJJAA2awf-latex-pseudocode-generator}{\LaTeX \ pseudocode generator}

\begin{algorithm}[H]
    \SetAlgoLined
    \SetKwFunction{spsolve}{spsolve}
    \KwIn{$\psi$ (initial wave function), $\Delta t$, $\Delta x$, $V$ (potential), $N_x$ (spatial resolution), $N_t$ (time steps)}
    \KwOut{$\psi_{\text{history}}$ (wave function at all time steps)}
    \BlankLine

    $\alpha \leftarrow i \Delta t / (4 \Delta x^2)$ \tcp*[r]{Precompute coefficient}
    $\mathbf{A} \leftarrow \text{ConstructMatrix}([- \alpha, 1 + 2\alpha + i \Delta t V / 2, - \alpha], [-1, 0, 1], N_x)$ \tcp*[r]{Left-hand matrix}
    $\mathbf{B} \leftarrow \text{ConstructMatrix}([\alpha, 1 - 2\alpha - i \Delta t V / 2, \alpha], [-1, 0, 1], N_x)$ \tcp*[r]{Right-hand matrix}

    % $\psi_{\text{history}} \leftarrow [\psi]$ \tcp*[r]{Record initial wave function}

    \For{$t \leftarrow 1$ \KwTo $N_t - 1$}{
        $\psi \leftarrow \spsolve(\mathbf{A}, \mathbf{B} \cdot \psi)$ \tcp*[r]{Solve $\mathbf{A} \psi^{(n+1)} = \mathbf{B} \psi^{(n)}$}
        Append $\psi$ to $\psi_{\text{history}}$ \tcp*[r]{Record updated wave function}
    }

    \Return $\psi_{\text{history}}$
    \caption{Crank-Nicolson Method for Time Evolution (Optimized)}
\end{algorithm}

\begin{algorithm}[H]
    \SetAlgoLined
    \SetKwFunction{spsolve}{spsolve}
    \SetKwFunction{roll}{np.roll}
    \KwIn{$\psi$ (initial wave function), $\Delta t$, $\Delta x$, $V$ (potential), $N_x$ (spatial resolution), $N_t$ (time steps)}
    \KwOut{$\psi_{\text{history}}$ (wave function at all time steps)}
    \BlankLine

    $\alpha \leftarrow i \Delta t / \Delta x^2, \psi_{\text{prev}} \leftarrow \psi, \psi_{\text{history}} \leftarrow [\psi]$ \tcp*[r]{Initialize constants and history}

    \SetKwFunction{CrankNicolsonStep}{CrankNicolsonStep}
    $\psi \leftarrow \CrankNicolsonStep(\psi, \Delta t, \Delta x, V)$ \tcp*[r]{Perform first step using CN method}
    Append $\psi$ to $\psi_{\text{history}}$ \tcp*[r]{Subsequent steps using explicit method}
    \For{$t \leftarrow 2$ \KwTo $N_t - 1$}{
        $\psi_{\text{current}} \leftarrow \psi, \psi_{\text{jp1}} \leftarrow \roll(\psi_{\text{current}}, 1), \psi_{\text{jm1}} \leftarrow \roll(\psi_{\text{current}}, -1)$ \tcp*[r]{Compute shifts}
        $\mathbf{L} \psi^n \leftarrow \psi_{\text{jp1}} + \psi_{\text{jm1}} - 2 \psi_{\text{current}}$ \tcp*[r]{Laplacian action}
        $\psi \leftarrow \psi_{\text{prev}} + \alpha \cdot (\mathbf{L} \psi^n) - 2i \Delta t V \cdot \psi_{\text{current}}$ \tcp*[r]{Update $\psi^{(n+1)}$}
        $\psi_{\text{prev}} \leftarrow \psi_{\text{current}}$ \tcp*[r]{Update $\psi^{(n-1)}$ for next step}
        Append $\psi$ to $\psi_{\text{history}}$ \tcp*[r]{Record updated wave function}
    }

    \Return $\psi_{\text{history}}$
    \caption{Explicit Time Evolution with Crank-Nicolson First Step (Optimized)}
\end{algorithm}

\subsection{结果示例}
\begin{figure}[H]
    \centering
    \includegraphics[width=1.0\textwidth]{Problem_2/figs/cn_result.png}
    \caption{Crank–Nicolson解法结果}
\end{figure}

\begin{figure}[H]
    \centering
    \includegraphics[width=1.0\textwidth]{Problem_2/figs/cn_anim.png}
    \caption{Crank–Nicolson解法中间态(动画截图)}
\end{figure}
本题使用的默认参数也可从图中读出,下图表明两种解法结果一致。动画运行需要先点击“显示动画”按钮,待动画生成就绪后会显示并自动播放,有进度条可供拖动回放。动画播放器存在一些已知bug,没力气修复了,不影响求解结果与静态图展示。
\begin{figure}[H]
    \centering
    \includegraphics[width=1.0\textwidth]{Problem_2/figs/ex_result.png}
    \caption{显式解法结果}
\end{figure}

\section{题目 3:变分法求解一维薛定谔方程}
\subsection{题目描述}
\noindent
Prove the stability condition of the explicit scheme of the 1D wave equation by performing Von Neumann stability analysis:
\[
    \frac{\partial^2 u}{\partial t^2} = c^2 \frac{\partial^2 u}{\partial x^2}.
\]
If \(c \Delta t / \Delta x \leq 1\), the explicit scheme is stable.

\subsection{证明}
使用中心差分离散化
\[
    \frac{u_{i,j+1} - 2u_{i,j} + u_{i,j-1}}{\Delta t^2} = c^2 \frac{u_{i+1,j} - 2u_{i,j} + u_{i-1,j}}{\Delta x^2}.
\]
令 \( \alpha = \frac{c \Delta t}{\Delta x} \),则可改写为:
\[
    u_{i,j+1} = 2u_{i,j} - u_{i,j-1} + \alpha^2 (u_{i+1,j} - 2u_{i,j} + u_{i-1,j}).
\]
假设数值解为傅里叶模式形式 \( u_{i,j} = \xi^j e^{iK i \Delta x} \),代入差分方程得到:
\[
    \xi^{j+1} e^{iK i \Delta x} = 2\xi^j e^{iK i \Delta x} - \xi^{j-1} e^{iK i \Delta x} + \alpha^2 \left( \xi^j e^{iK (i+1) \Delta x} - 2\xi^j e^{iK i \Delta x} + \xi^j e^{iK (i-1) \Delta x} \right).
\]
化简得到
\[
    \xi - 2 + 1/\xi = \alpha^2 (e^{iK \Delta x} + e^{-iK \Delta x} - 2) = -4\alpha^2 \sin^2\left(\frac{K \Delta x}{2}\right).
\]
设 \( \beta = 1 - 2\alpha^2 \sin^2\left(\frac{K \Delta x}{2}\right) \),方程化为:
\[
    \xi^2 - 2\beta \xi + 1 = 0,
\]
其解为:
\[
    \xi = \beta \pm \sqrt{\beta^2 - 1}.
\]
根据冯·诺伊曼稳定性分析,为了数值方案稳定,要求放大因子 \( \xi \) 的模满足 \( |\xi| \leq 1 \).由于 \( \beta = 1 - 2\alpha^2 \sin^2\left(\frac{K \Delta x}{2}\right) \),当 \( |\beta| \leq 1 \) 时,有 \( \beta^2 - 1 \leq 0 \),此时 \( \xi = \beta \pm i\sqrt{1 - \beta^2} \),模长恰好为:
\[
    |\xi| = \sqrt{\beta^2 + (1 - \beta^2)} = 1.
\]
而一旦 \( |\beta| > 1 \),则 \( |\xi_{+}| > 1 \),数值方案不稳定。因此,为了 \( |\xi| \leq 1 \),需满足
\[
    0\leq\alpha^2\sin^2(\frac{K\Delta x}2)\leq1, \quad \forall K \in \mathbb{R}.
\]
亦即要求
\[
    \boxed{\alpha = \frac{c \Delta t}{\Delta x} \leq 1}
\]
此时一维波动方程的显式差分格式是稳定的。

\vspace{5pt}
\end{document}