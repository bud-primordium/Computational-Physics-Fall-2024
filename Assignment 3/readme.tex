\documentclass[11pt]{article}
\usepackage[top=1cm, bottom=2cm, left=1cm, right=1cm]{geometry}
\usepackage{ctex}
\usepackage[linesnumbered,ruled]{algorithm2e}
\usepackage{amsthm,amsmath,amssymb}
\usepackage[colorlinks=true,linkcolor=blue]{hyperref}
\usepackage{listings}
\usepackage{xcolor,xparse}
\usepackage{realboxes}
\usepackage{mathrsfs}
\usepackage{wrapfig}
\usepackage{subfigure}
\usepackage{forest}
\usepackage{pifont}
\usepackage{bm}

\SetKw{Print}{print}
\SetKw{Read}{read}
\SetKw{Allocate}{allocate}
\SetKw{Deallocate}{deallocate}
\SetKw{Call}{call}
\SetKw{True}{true}
\SetKw{False}{false}
\SetKw{Continue}{continue}
\SetKw{Break}{break}
\SetKw{Write}{write}

\definecolor{cmdbg}{rgb}{0.9,0.9,0.9}
\lstset{%
	basicstyle=\ttfamily,
	breaklines = true,
	backgroundcolor=\color{cmdbg},
}
\DeclareDocumentCommand{\ccmd}{v}{% 参数 v 表示工作方法类似于 \verb
    \Colorbox{cmdbg}{\csname lstinline\endcsname!#1!}%
}


\author{杨远青 22300190015}
\title{计算物理作业3}

\begin{document}
\maketitle
\textit{远方来朋,喜;假期俱至,悦。}
\section{题目 1:高斯消元法的时间复杂度分析}

\subsection{题目描述}
\noindent
Consider the Poisson equation:
\[
    \nabla^2 \varphi(x, y) = -\frac{\rho(x, y)}{\varepsilon_0}
\]
from electrostatics on a rectangular geometry with \(x \in [0, L_x]\) and \(y \in [0, L_y]\). Write a program that solves this equation using the relaxation method and test your program with the following cases:

\noindent
(a) \(\rho(x, y) = 0\), \(\varphi(0, y) = \varphi(L_x, y) = \varphi(x, 0) = 0\), \(\varphi(x, L_y) = 1 \, \text{V}\),
\(L_x = 1 \, \text{m}\), and \(L_y = 1.5 \, \text{m}\);

\noindent
(b) \(\frac{\rho(x, y)}{\varepsilon_0} = 1 \, \text{V/m}^2\), \(\varphi(0, y) = \varphi(L_x, y) = \varphi(x, 0) = \varphi(x, L_y) = 0\), and \(L_x = L_y = 1 \, \text{m}\).


\subsection{程序描述}

\subsection{伪代码}
Powered by \href{https://chatgpt.com/g/g-xJJAA2awf-latex-pseudocode-generator}{\LaTeX \ pseudocode generator}


\subsection{结果示例}


\vspace{5pt}
\end{document}