\subsection{题目描述}
Prove that the time complexity of $\bm{LU}$ decomposition algorithm is $\mathcal{O}(n^3)$.
\subsection{证明}
$\bm{LU}$分解法的第一步是将系数矩阵$\bm{A}$分解为一个下三角矩阵$\bm{L}$和一个上三角矩阵$\bm{U}$:
\[
    \begin{bmatrix}
        a_{11} & a_{12} & \cdots & a_{1n} \\
        a_{21} & a_{22} & \cdots & a_{2n} \\
        \vdots & \vdots & \ddots & \vdots \\
        a_{n1} & a_{n2} & \cdots & a_{nn}
    \end{bmatrix}
    =
    \begin{bmatrix}
        l_{11} & 0      & \cdots & 0      \\
        l_{21} & l_{22} & \cdots & 0      \\
        \vdots & \vdots & \ddots & \vdots \\
        l_{n1} & l_{n2} & \cdots & l_{nn}
    \end{bmatrix}
    \begin{bmatrix}
        1      & u_{12} & \cdots & u_{1n} \\
        0      & 1      & \cdots & u_{2n} \\
        \vdots & \vdots & \ddots & \vdots \\
        0      & 0      & \cdots & 1
    \end{bmatrix}.
\]
这一步常采用Crout方法实现,即在每一轮中,我们先计算$\bm{L}$的第$k$列元素$l_{ik}$,
\[
    l_{ik} = a_{ik} - \sum_{s=1}^{k-1} l_{is} u_{sk}, \quad \ i = k, k+1, \dots, n.
\]
每一个$l_{ik}$的计算涉及$k-1$次乘法和$k-1$次减法,共有$(n-k+1)$个$l_{ik}$需要计算;再计算$\bm{U}$的第$k$行元素$u_{kj}$,
\[
    u_{kj} = \frac{1}{l_{kk}} \left( a_{kj} - \sum_{s=1}^{k-1} l_{ks} u_{sj} \right), \quad \ j = k+1, k+2, \dots, n.
\]
相比$l_{ik}$的计算多了一次除法,共有$(n-k)$个$u_{kj}$需要计算,故第$k$轮的操作数为
\[
    (n-k+1)\cdot(2k-2)+(n-k)\cdot(2k-1) = -4k^2 + (4n+5)k -3n -2.
\]
因此,分解步骤的总操作数为
\[
    T_c(n) =\sum_{k=1}^{n} \ [-4k^2 + (4n+5)k -3n -2]                                                 \\
    = -4 \cdot \frac{n(n+1)(2n+1)}{6} + (4n+5) \cdot \frac{n(n+1)}{2} - (3n+2) \cdot n\\
    = \frac{4n^3 - 3n^2 - n}{6}.
\]

再考虑回代步骤的操作数,即用分解得到的$\bm{L}$和$\bm{U}$求解方程组$\bm{Ax} = \bm{b}$。首先求解$\bm{Ly} = \bm{b}$,即
\[
    \begin{bmatrix}
        l_{11} & 0      & \cdots & 0      \\
        l_{21} & l_{22} & \cdots & 0      \\
        \vdots & \vdots & \ddots & \vdots \\
        l_{n1} & l_{n2} & \cdots & l_{nn}
    \end{bmatrix}
    \cdot
    \begin{bmatrix}
        y_1 \\ y_2 \\ \vdots \\ y_n
    \end{bmatrix}
    =
    \begin{bmatrix}
        b_1 \\ b_2 \\ \vdots \\ b_n
    \end{bmatrix}.
\]
这实质上是从第一行开始的\textit{Forward Substitution},即
\[
    y_i=\frac{1}{l_{i i}}\left(b_i-\sum_{j=1}^{i-1} l_{i j} y_j\right).
\]
每一步有$1$次除法,$(i-1)$次乘法与$(i-1)$次减法;再求解$\bm{Ux} = \bm{y}$,即
\[
    \begin{bmatrix}
        1      & u_{12} & \cdots & u_{1n} \\
        0      & 1      & \cdots & u_{2n} \\
        \vdots & \vdots & \ddots & \vdots \\
        0      & 0      & \cdots & 1
    \end{bmatrix}
    \cdot
    \begin{bmatrix}
        x_1 \\ x_2 \\ \vdots \\ x_n
    \end{bmatrix}
    =
    \begin{bmatrix}
        y_1 \\ y_2 \\ \vdots \\ y_n
    \end{bmatrix},
\]
这实质上是从最后一行开始的\textit{Backward Substitution},即
\[
    x_i=\left(y_i-\sum_{j=i+1}^{n} u_{i j} x_j\right).
\]
每一步有$(n-i)$次乘法与$(n-i)$次减法,故回代步骤操作数为
\[
    T_s(n)=\sum_{i=1}^{n} \ [(2i-1)+(2n-2i)] = \sum_{i=1}^{n} (2n-1) = n(2n-1) = 2n^2 -n.
\]
因此,$\bm{LU}$分解法的总操作数为
\[
    T(n)=T_c(n)+T_s(n)=\frac{4n^3 -3n^2 -n}{6} + 2n^2 -n = \frac{4n^3 +9n^2 -7n}{6}.
\]
其中有除法$n(n+1)/2$次,乘法与减法各$n(n-1)(2n+5)/6$次,故
\[
    \boxed{T(n) = \mathcal{O}(n^3)}
\]
\textit{Amazing,居然与Gaussian消元法的各种操作数都相同!}